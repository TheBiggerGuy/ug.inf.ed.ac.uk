\documentclass[11pt,twoside,a4paper]{article}

\usepackage{hyperref}
\usepackage{fixltx2e}
\usepackage{amsmath}

% \textsubscript{}
% \textsuperscript{}

\begin{document}
  
  \title{Database Systems \\Revision Notes}
  \author{Guy Taylor}
  \date{April 2011}
  
  \maketitle
  
  \tableofcontents
  
  \clearpage
  
  \section{Introduction}
    This document is a set of revision notes for the Database Systems \cite{dbs_home} course at the Univerisy of Edinbugh.
  
  \clearpage
  \section{Relational Algebra}
    Relatinal algebra is syntax and set of rules that allows mathimatical like operations to be applied to a relatinal database.
    
    \subsection{Selection}
      Selection ($\sigma$) is a method simmilar to SELECT in SQL. It allows a data set to be filtered by a, or several, logical exspresions.
    
      \begin{tabular}{lcr}
        % start left
      	\begin{tabular}{l|l|l}
      		Name & Age & Sex \\
      		\hline
      		Bob  & 22  & M \\
      		Bill & 32  & M \\
      		Ben  & 42  & F \scriptsize{(dont ask)}\\
      	\end{tabular}
        % end left
      	&
      	% start center
      	$ \sigma_{(age > 30)}(People) = $
      	% end center
      	&
      	% start right
      	\begin{tabular}{c|l|c}
      		Name & Age & Sex \\
      		\hline
      		Bill & 32  & M \\
      		Ben  & 42  & F \\
      	\end{tabular}
      	% end right
      \end{tabular}
    
    \subsection{Projection}
      Projection ($\mathrm{\pi}$) us a method to filter colums out of a data set.
      
      \begin{tabular}{lcr}
        % start left
      	\begin{tabular}{l|l|l}
      		Name & Age & Sex \\
      		\hline
      		Bob  & 22  & M \\
      		Bill & 32  & M \\
      		Ben  & 42  & F \scriptsize{(dont ask)}\\
      	\end{tabular}
        % end left
      	&
      	% start center
      	$ \pi_{(name,age)}(People) = $
      	% end center
      	&
      	% start right
      	\begin{tabular}{c|l|c}
      		Name & Age \\
      		\hline
      		Bill & 32\\
      		Ben  & 42\\
      	\end{tabular}
      	% end right
      \end{tabular}
  
  \clearpage
  % \addcontentsline{toc}{chapter}{Bibliography}
  
  \bibliography{main}
  \bibliographystyle{apalike}

\end{document}
