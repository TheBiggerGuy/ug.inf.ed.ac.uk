\documentclass[11pt,twoside,a4paper]{article}

\usepackage{hyperref}
\usepackage{fixltx2e}
\usepackage{amsmath}

% \textsubscript{}
% \textsuperscript{}

\begin{document}
  
  \title{Database Systems \\Revision Notes}
  \author{Guy Taylor}
  \date{April 2011}
  
  \maketitle
  
  \tableofcontents
  
  \clearpage
  
  \section{Introduction}
    This document is a set of revision notes for the Database Systems\textsuperscript{\cite{dbs_home}} course at the Univerisy of Edinbugh.
  
  \clearpage
  \section{Relational Algebra}
    Relatinal algebra is syntax and set of rules that allows mathimatical like operations to be applied to a relatinal database.
    
    \subsection{Selection}
      Selection ($\sigma$) is a method simmilar to SELECT in SQL. It allows a data set to be filtered by a, or several, logical exspresions.
    
      \begin{tabular}{lcr}
        % start left
      	\begin{tabular}{l|l|l}
      		Name & Age & Sex \\
      		\hline
      		Bob  & 22  & M \\
      		Bill & 32  & M \\
      		Ben  & 42  & F\textsuperscript{\cite{dont_ask}}\\
      	\end{tabular}
        % end left
      	&
      	% start center
      	$ \sigma_{(age > 30)}(People) $
      	% end center
      	&
      	% start right
      	\begin{tabular}{c|l|c}
      		Name & Age & Sex \\
      		\hline
      		Bill & 32  & M \\
      		Ben  & 42  & F \\
      	\end{tabular}
      	% end right
      \end{tabular}
    
    \subsection{Projection}
      Projection ($\mathrm{\pi}$) us a method to filter colums out of a data set.
      
      \begin{tabular}{lcr}
        % start left
      	\begin{tabular}{l|l|l}
      		Name & Age & Sex \\
      		\hline
      		Bob  & 22  & M \\
      		Bill & 32  & M \\
      		Ben  & 42  & F \\
      	\end{tabular}
        % end left
      	&
      	% start center
      	$ \pi_{(name,age)}(People) $
      	% end center
      	&
      	% start right
      	\begin{tabular}{c|l}
      		Name & Age \\
      		\hline
      		Bill & 32\\
      		Ben  & 42\\
      	\end{tabular}
      	% end right
      \end{tabular}
    
    \clearpage
    \section{SQL}
    	\subsection{Create}
    	  \begin{tabbing}
    	    CREATE \= TABLE \emph{table1}\\
    	    \>(                           \\
      		\>\emph{name1} \emph{type1},  \\
      		\>\emph{name2} \emph{type2},  \\
      		\>\emph{name3} \emph{type3},  \\
      		\>\emph{name4} \emph{type4},  \\
      		\>PRIMARY KEY (\emph{name1}), \\
      		\>FOREIGN KEY (\emph{name2}) REFERENCES \emph{table2} (\emph{name1}) \\
      		\>);
      	\end{tabbing}
     
     \subsection{Select}
    	  \begin{tabbing}
    	    SELECT \= \\
    	    \>name1, name2                \\
    	    FROM                          \\
    	    \>table1                      \\
    	    WHERE                         \\
    	    \>name1 = name3               \\
      		;
      	\end{tabbing}
    
    \subsection{Insert}
    	  \begin{tabbing}
    	    INSTER \= INTO                \\
    	    \>table1 (name1, name2)       \\
    	    VALUES                        \\
    	    \>(value1, value2)            \\
      		;
      	\end{tabbing}
     
     \subsection{Update}
    	  \begin{tabbing}
    	    UPDATE \=                    \\
    	    \>table1                      \\
    	    SET                           \\
    	    \>name1=value1, name2=value2  \\
    	    WHERE                         \\
    	    \>name3 = value3              \\
      		;
      	\end{tabbing}
    		  
  
  \clearpage  
  \begin{thebibliography}{9}
    \bibitem{dbs_home}
      "Database Systems Home page" by "Peter Buneman" \\
      "\url{http://homepages.inf.ed.ac.uk/opb/dbs/}"
    \bibitem{dont_ask}
      Dont ask
  \end{thebibliography}

\end{document}
