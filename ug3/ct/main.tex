\documentclass[11pt,twoside,a4paper]{article}

\usepackage{hyperref}

\begin{document}
  
  \title{Compiling Techniques \\Revision Notes}
  \author{Guy Taylor}
  \date{April 2011}
  
  \maketitle
  
  \tableofcontents
  
  \section{Introduction}
    This document is a set of revision notes for the Compiling Techniques\textsuperscript{\cite{ct_home}} course at the Univerisy of Edinbugh.
  
  \clearpage
  \section{Parsing}
    Parsing
    
    \begin{center}\begin{tabular}{l|l|l}
       & Top Down & Bottom Up \\ \hline
      Design & TODO & Can create a state machine from the grammer \\
      Efficent & TODO & Efficent \\
      Method & TODO & Start resurse; replace all NT; untill no NT (None Terminal) left; \\
    \end{tabular}\end{center}
    
    \subsection{Bottom UP}
      \subsubsection{Reduction}
      As u work backwards you get the parentless leafs. These are the 'upper fringe'. You then parse a new leafe untill they join. This is called reduction. \\
      Side Note: 'Finding Reductions' position is referedfrom the most right char possition.
      
      \subsubsection{Hurdles}
        Find the right hand side that is a definate terminal (ie ; in c) \\
        Allows stuff to be streamed efficently (?? maybe ??)
      
      \subsubsection{Shift-Reduce}
        Bad errors: Errors can allways be found but as there is a laxck of contect no good error message can be produced
      
      \subsubsection{LR(1)}
        \begin{itemize}
          \item Bottom up
          \item table based
          \item Shift-replace
          \item one char look ahead
        \end{itemize}
        Extra: There can be more than one char look ahead (ie. LR(2), LR(3) ...) \\
        Note: Although the language is contex free we use DFA (D.. Finite Atomata) to prosses it as it is only done 'line by line'.
  
  \clearpage  
  \begin{thebibliography}{9}
    \bibitem{ct_home}
      "Compiler Techinques Home page" by "Björn Franke", "\url{http://www.inf.ed.ac.uk/teaching/courses/ct/}"
  \end{thebibliography}

\end{document}
